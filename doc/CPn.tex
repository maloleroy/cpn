\documentclass[en]{article}

\usepackage{amsfonts}

\title{CPn: C polynomials}
\author{M. Leroy}

\begin{document}

\maketitle

\section{Abstract}

The purpose of this program is to perform operations and calculations on polynomials. This includes adding, substracting, multiplying or dividing (as in Euclidean division), or finding a greatest common divisor. \\

\textit{In this paragraph, T refers to a mathematical set, like $\mathbb{Q}$ or $\mathbb{Z}[X]$.} \\

If an object uses dynamic memory allocation, it is created by a \texttt{make\_T} function. In this case, we should call \texttt{free\_T} on this object before ending the program. If at the opposite an object does not use memory allocation, but stil needs to be created by a function, this function should be named \texttt{new\_T}. We do not need to free the memory used by objects created by such a function.

\section{Typing}

All polynomials must (mathematically) have their coefficients in $\mathbb{Q}$, since we do not introduce irrational coefficients (at least for now). The typing respects the usual operations in sub-rings ($(\mathbb{Z}[X], +, \cdot)$ and $(\mathbb{Q}[X], +, \cdot)$):
\begin{itemize}
    \item $\forall (P, Q) \in \mathbb{Z}[X]^2, P + Q \in \mathbb{Z}[X]$
    \item $\forall (P, Q) \in \mathbb{Z}[X]^2, PQ \in \mathbb{Z}[X]$
    \item $\forall (P, Q) \in \mathbb{Q}[X]^2, P + Q \in \mathbb{Q}[X]$
    \item $\forall (P, Q) \in \mathbb{Q}[X]^2, PQ \in \mathbb{Q}[X]$
\end{itemize}
Besides that, there is an automatic upward type conversion:
\begin{itemize}
    \item $\forall (P, Q) \in \mathbb{Z}[X] \times \mathbb{Q}[X], P + Q \in \mathbb{Q}[X]$
    \item $\forall (P, Q) \in \mathbb{Z}[X] \times \mathbb{Q}[X], PQ \in \mathbb{Q}[X]$
\end{itemize}

\subsection{Fast polynomials}

\begin{itemize}
    \item With integer coefficients ($\mathbb{Z}[X]$)
    \item With rational coefficients ($\mathbb{Q}[X]$)
\end{itemize}

\subsection{Named polynomials}


\end{document}
